%% vim:tw=66:spell:wrap:ft=tex:fdm=marker
% Preamble {{{
\documentclass[%
        hyperref={%
                pdfauthor={Zakariyya Mughal},%
                pdfpagemode={None},pdfpagelayout={SinglePage}}%
        xcolor={x11names},%
]{beamer}
\usetheme{Warsaw}
\usecolortheme{beaver}
\usepackage{textcomp}
\usepackage{fancyvrb}
\usepackage{changepage}
\usepackage{multicol}
\usepackage{wasysym}
\usepackage[T1]{fontenc}
\usepackage{listings}

\lstset{%basicstyle=\small\ttfamily,
%numbers=left,
%escapeinside=||
}
\newenvironment{indented}{\begin{adjustwidth}{1.5em}{}}{\end{adjustwidth}}

\usepackage{tikz}
\usetikzlibrary{snakes,arrows,shapes,automata}

\title[Unix Scripting]{Unix workshop: Let's Go Scripting}
\author{Zaki Mughal}
\institute{Computational Biomedicine Lab\\University of Houston}
\date{2013 Nov 20}
%}}}
\begin{document}

\frame{\titlepage}
\begin{frame}
	Slides are at \url{http://bit.ly/csgsa_unix_f2013}.
\end{frame}
\section{Introduction}\frame{\insertsection}

\begin{frame}
	\frametitle{This time}
	\begin{itemize}
		\item processes
		\item more on I/O redirection
		\item screen (terminal multiplexer)
		\item advanced scripting
		\item network
	\end{itemize}
\end{frame}
% Processes {{{
\section{Processes}\frame{\insertsection}
\begin{frame}
	\frametitle{Concepts}
	\begin{itemize}
		\item process ID (PID) --- associated with each
			process
		\item \texttt{init} --- first process that starts when you
			boot (PID: 1)
		\item \texttt{pstree} --- see tree of processes
			(\texttt{fork}ed off)
	\end{itemize}
\end{frame}
\begin{frame}
	\frametitle{Job control}
	\begin{itemize}
		\item when you run a process it is in \emph{foreground}
		\item Ctrl+C --- will interrupt the process
		\item Ctrl+Z --- will suspend the process
		\pause\item these are shell functions
		\pause\item \texttt{bg} (shell builtin to background
			process)
		\pause\item \texttt{fg} (and foreground)
		\pause\item \texttt{jobs} (and see the processes under
			job control) (fg \%1, bg \%2)
	\end{itemize}
\end{frame}
\begin{frame}
	\frametitle{pgrep}
	\begin{itemize}
		\item you can find the process ID using:
		\item \texttt{pstree -p}
		\item \texttt{ps} (my processes in this terminal)
		\item \texttt{ps -u} (my processes)
		\item \texttt{ps -elf} (everyone's processes)
		\pause\item \texttt{pgrep}
		\item \texttt{pgrep -lf}
		\item the name will make sense more sense in a bit
	\end{itemize}
\end{frame}
\begin{frame}
	\frametitle{Signals}
	\begin{itemize}
		\item internally, they send signals to the process
		\item Ctrl+C --- SIGINT
		\item Ctrl+Z --- SIGTSTP
		\pause\item \texttt{man 7 signal \# to find out more}
		\item also read \url{http://linusakesson.net/programming/tty/}
	\end{itemize}
\end{frame}
\begin{frame}
	\frametitle{Signals}
	\begin{itemize}
		\item you can send signals using \texttt{kill}
		\pause\item more types of signals
		\pause\item SIGTERM, SIGKILL \pause (SIGKILL can not be caught by process)
		\pause\item numbers associated (\texttt{kill -l})
		\pause\item SIGSEGV
		\item	\texttt{int main() \{ char* a = 0; printf(*a); return 0; \}}
					
	\end{itemize}
\end{frame}
\begin{frame}
	\frametitle{Signals using kill}
	\begin{itemize}
		\item kill [pid] \# default to send SIGTERM (15)
		\pause\item kill -TERM [pid] \# same as default
		\pause\item \texttt{kill -9 [pid]} \# send SIGKILL (DANGER)
	\end{itemize}
\end{frame}
\begin{frame}
	\frametitle{Exit status}
	\begin{itemize}
		\item what happens when I kill?
		\pause\item \texttt{echo \$?}
		\pause\item 0 indicates success \pause (int main \{ \ldots return 0; \} ?)
			\pause (\texttt{return EXIT\_SUCCESS}, \texttt{return EXIT\_FAILURE})
		\pause\item \texttt{\$?} is a environment variable (more later)
		\pause\item after \texttt{kill -TERM}, we get 143
		\pause\item 128 + 15 (the number for SIGTERM)
	\end{itemize}
\end{frame}
\begin{frame}
	\frametitle{Multiple commands}
	\begin{itemize}
		\item \texttt{cat file1 ; cat file2}
		\pause\item these could fail (file1 does not exist)

		\pause\item \texttt{cat file1 \&\& cat file2}
		\pause\item only run when the first is successful

		\pause\item \texttt{cat file1 || echo "could not cat!"}
		\pause\item only run second when the first is NOT successful
	\end{itemize}
\end{frame}
\begin{frame}
	\frametitle{Sleep}
	\begin{itemize}
		\item \texttt{date \&\& sleep 5m \&\& echo "Nap
		time is over!" \&\& date }
		\pause\item BONUS COMMAND: \texttt{date}
	\end{itemize}
\end{frame}
\begin{frame}
	\frametitle{Where are all these binaries anyway?}
	\begin{itemize}
		\item \texttt{echo \$PATH}
		\item another environment variable
		\pause\item colon-separated paths
		\item which ls
		\item which vi
		\item which -a matlab
	\end{itemize}
\end{frame}
\begin{frame}
	\frametitle{A taste of scripting!}
	\begin{itemize}
		\item \texttt{vi \textasciitilde/.bashrc}
		\item \texttt{alias ..='cd ..'}
		\item \texttt{alias l='ls -CF'}
		\item \texttt{alias c='cd'}
		\item shortcuts!
		\pause\item disable with a backslash:
		\item \textbackslash{}ls
	\end{itemize}
\end{frame}
\begin{frame}
	\frametitle{A taste of scripting!}
	\begin{itemize}
		\item you can set environment variables here
		\item set the \$PATH and other environment variables
		\item export PATH=~/script:\$PATH \# prepend
		\item export EDITOR=vim
		\item export PAGER=less
		\pause\item speaking of paths, spaces are bad for scripting --- don't use spaces in your filenames
	\end{itemize}
\end{frame}
%}}}

% I/O Redirection {{{
\section{I/O redirection}\frame{\insertsection}
\begin{frame}
	\frametitle{File redirection}
	\begin{itemize}
		\item which ls \&\& echo "ls is available"
		\pause\item extra output
		\pause\item which ls > /dev/null \&\& echo "ls is available"
		\pause\item stdout goes to /dev/null
	\end{itemize}
\end{frame}
\begin{frame}
	\frametitle{Back to cat}
	\begin{itemize}
		\item Remember \texttt{cat file}
		\pause\item \texttt{cat}
		\item read from stdin
		\item \ldots
		\pause\item hit Ctrl+D (sends EOF, zero-bytes left to
			read)
		\pause\item then outputs the stdin to stdout
	\end{itemize}
\end{frame}
\begin{frame}
	\frametitle{Make every program a filter}
	\begin{itemize}
		\item \texttt{cat file} is equivalent to:
		\item \texttt{cat < file}
		\item many programs follow this principle
		\pause\item \texttt{wc}
		\item word count
		\pause\item \texttt{wc -l < essay.txt}
		\item how many lines?
	\end{itemize}
\end{frame}
\begin{frame}
	\frametitle{Make every program a filter}
	\begin{itemize}
		\item shuf (shuffle the files)
		\item grep (match using regex)
		\item less (navigate long output)
		\item head, tail (see n lines of start/end of input)
		\item sort
		\item uniq (unique lines)
	\end{itemize}
\end{frame}
\begin{frame}[fragile]
	\frametitle{Pipes}
	\begin{itemize}
		\item instead of files, send output of one command
			to the output of another
		\item let's look at some examples:

		\item \begin{lstlisting}
			ls | grep -o '\.[a-z][^.]*$' | sort | uniq
			\end{lstlisting}
		\pause\item \begin{lstlisting}
			ls | grep -o '\.[a-z][^.]*$' \
				| sort | uniq \
				| wc -l
			\end{lstlisting}
	\end{itemize}
\end{frame}
\begin{frame}[fragile]
	\frametitle{Pipes}
	\begin{itemize}
		\item List all file extensions (starting with a--z)
\begin{lstlisting}
ls | grep -o '\.[a-z][^.]*$' | sort | uniq
\end{lstlisting}
		\pause\item How many of them?
\begin{lstlisting}
ls | grep -o '\.[a-z][^.]*$' \
   | sort | uniq \
   | wc -l
\end{lstlisting}
		\item we're using the backslash for line-continuation
	\end{itemize}
\end{frame}
\begin{frame}[fragile]
	\frametitle{Pipes}
	\begin{itemize}
		\item Which ones are there the most of?
\begin{lstlisting}
ls | grep -o '\.[a-z][^.]*$' \
   | sort    | uniq -c \
   | sort -n | tail
\end{lstlisting}
	\end{itemize}
\end{frame}
%}}}


% Screen {{{
\section{Screen}\frame{\insertsection}
\begin{frame}
	\begin{itemize}
\item GNU Screen (you may have to install it)
\item terminal multiplexer --- multiple terminals in one terminal
\item persistent session
\pause\item \texttt{screen -S session\_name}
\item Ctrl+A (the command prefix)
\item Ctrl+A a (actual Ctrl+A)
\item Ctrl+A c (create a new window)
\item Ctrl+A n (next window)
\item Ctrl+A p (previous window)
\item Ctrl+A d (detach)
\item \texttt{screen -d -r session\_name \# reattach}
\item Ctrl+A ? (help)
	\end{itemize}
\end{frame}
\begin{frame}
\begin{itemize}
\item Ctrl+A S (splits into regions [horizontal])
\item Ctrl+A Tab (move between regions)
\item Ctrl+A | (splits into regions [vertical])

\pause\item NOT Ctrl+A s !!!!
\item that will stop flow control to the terminal
\item \qquad	fix by doing Ctrl+A q
\pause\item similar to Ctrl+S, Ctrl+q in terminal itself
\item XOFF/XON, see \url{http://unix.stackexchange.com/questions/12107/}, \url{https://en.wikipedia.org/wiki/Software_flow_control}
\end{itemize}
\end{frame}
%}}}

% Scripting {{{
\section{Scripting}\frame{\insertsection}
\begin{frame}[fragile]
	\begin{itemize}
		\item Script files begin with the line (shebang)
		\item
\begin{lstlisting}
#!/bin/sh
\end{lstlisting}
	\item chmod u+x myscript.sh \# make executable
	\item ./myscript.sh \# run it
	\end{itemize}
\end{frame}
\begin{frame}[fragile]
	\begin{itemize}
		\item we can put any of the commands we used
			before
		\item \texttt{count\_extensions.sh}
		\item fc (open last command in editor)
		\item Ctrl+X Ctrl+C (open current command in editor)
	\end{itemize}
\end{frame}
\begin{frame}[fragile]
	\frametitle{Command substitution}
	\begin{itemize}
		\item
		\begin{lstlisting}
			ls -l \$(which ls)
		\end{lstlisting}
		\item 
		\begin{lstlisting}
		ls -l `which ls` # backticks (same key as tilde on US QWERTY)
		\end{lstlisting}
	\item find the location of \texttt{ls} and give
	\end{itemize}
\end{frame}
\begin{frame}[fragile]
	\frametitle{Writing a more complex script}
	\begin{itemize}
		\item watch ls \# repeatedly run command
		\pause\item write your own watch (well, slightly
			modified)
	\end{itemize}
\end{frame}
\begin{frame}[fragile]
	\frametitle{Writing a more complex script}
	\begin{lstlisting}[escapeinside=++]
	#!/bin/sh
	if ! which sleep clear > /dev/null; then
	  # check that sleep and clear are in path
	  echo "Require: sleep, clear"
	  exit 1
	fi
	+\pause+while [ "$?" = 0 ]; do
	  # man 1 test # (test for files, dirs, etc.)
	  # run the args: "$1", "$2"
	  "$@" \
	     && sleep 2 \
	     && clear # clear the screen
	done
	\end{lstlisting}
\end{frame}
\begin{frame}[fragile]
	\frametitle{Managing lots of files}
	\begin{lstlisting}
	for i in `ls`; do
	  echo "$i";
	done
	\end{lstlisting}
\end{frame}
\begin{frame}[fragile]
	\frametitle{Managing lots of files}
	\begin{lstlisting}
	for i in `ls`; do
	  if [ -f "$i" ] \
	      && echo "$i" | grep -iq ".tex$"; then
	    # check if it is a file
	    # and that it ends in tex
	    echo "Found a TeX file: $i";
	  fi
	done
	\end{lstlisting}
\end{frame}
\begin{frame}[fragile]
	\frametitle{Managing lots of files}
	\begin{lstlisting}
		find -type f -iname '*.tex'
		# same but does it recursively
	\end{lstlisting}
\end{frame}
\begin{frame}[fragile]
	\frametitle{Managing lots of files}
	\begin{lstlisting}
		find -type f -iname '*.tex' \
		  -exec echo "Found a TeX file: {}" \;
	\end{lstlisting}
\end{frame}
\begin{frame}[fragile]
	\frametitle{Managing lots of files}
	\begin{lstlisting}
		# the most robust way
		# -print0 : null separated
		find -type f -iname '*.tex' -print0 \
		  | xargs -I{} -0 echo "Found a TeX file: {}"
		# xargs takes stdin and passes it as an argument
		# to a command
	\end{lstlisting}
\end{frame}

			
\begin{frame}
	More info at the Advanced Bash-Scripting Guide: \url{http://tldp.org/LDP/abs/html/}
\end{frame}
%}}}

% Network {{{
\section{Network}\frame{\insertsection}
\begin{frame}[fragile]
	\begin{itemize}
\item ssh, sftp --- remote terminal and download files
\item \begin{lstlisting}
		diff <(ssh FAR_AWAY ls ~/sw_projects) \
		     <(ls ~/sw_projects)
		# find out which files are different
		# in a remote directory
		# uses process substitution <() (mkfifo)
		# + ssh
		# + diff
	\end{lstlisting}
\pause\item sshfs --- mount remote computers as a file system
\pause\item rsync, unison --- unidirectional and bidirectional backups/syncing
\pause\item wget --- mirror FTP/HTTP
	\end{itemize}
\end{frame}
%}}}

\end{document}
