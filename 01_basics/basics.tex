%% vim:tw=66:spell:wrap:ft=tex:fdm=marker
% Preamble {{{
\documentclass[%
        hyperref={%
                pdfauthor={Zakariyya Mughal},%
                pdfpagemode={None},pdfpagelayout={SinglePage}}%
        xcolor={x11names},%
]{beamer}
\usetheme{Warsaw}
\usecolortheme{beaver}
\usepackage{textcomp}
\usepackage{fancyvrb}
\usepackage{changepage}
\usepackage{multicol}
\usepackage{wasysym}
\usepackage[T1]{fontenc}
\usepackage{listings}

\lstset{%basicstyle=\small\ttfamily,
%numbers=left,
%escapeinside=||
}
\newenvironment{indented}{\begin{adjustwidth}{1.5em}{}}{\end{adjustwidth}}

\usepackage{tikz}
\usetikzlibrary{snakes,arrows,shapes,automata}

\title[Unix Basics]{Unix workshop: The Basics}
\author{Zaki Mughal}
\institute{Computational Biomedicine Lab\\University of Houston}
\date{2013 Nov 13}
%}}}
\begin{document}

\frame{\titlepage}
\begin{frame}
	Slides are at \url{http://bit.ly/csgsa_unix_f2013}.
\end{frame}
\section{Introduction}\frame{\insertsection}
% intro {{{
\begin{frame}
	\frametitle{Intro}
	\begin{itemize}
		\item Unix : operating system developed at Bell
			Labs in 1969
		\pause\item written in C and assembly \\
			\pause \qquad portability
		\pause\item response to Multics OS project
		\pause \item developed officially for text processing \pause (but
			really games)
			% c.f. roff, <https://en.wikipedia.org/wiki/Space_Travel_(video_game)>
		\pause \item multi-user (time-sharing)
		\pause \item distributed to anyone that wanted it
	\end{itemize}
\end{frame}

\begin{frame}
	\frametitle{Later}
	\begin{itemize}
		\item BSD $\rightarrow$ FreeBSD, NetBSD, OpenBSD
		\pause \item HP-UX, Solaris, AIX, Irix, Xenix
		\pause \item \emph{Linux}
		\pause \item Mac OS X
	\end{itemize}

	\pause POSIX, SUS --- standardisation of API and environment between
	Unix systems
\end{frame}

\begin{frame}
	\frametitle{Concepts}
	\begin{itemize}
		\item (almost) everything is a file
		\item single tree (no concepts of multiple drives)
		\item top is \texttt{/} (root directory)
	\end{itemize}
\end{frame}

\begin{frame}
	\frametitle{Standard directories (where's the thing for that thing)}
	% <https://en.wikipedia.org/wiki/Unix_directory_structure#Conventional_directory_layout>
	% for Linux, see <https://en.wikipedia.org/wiki/Filesystem_Hierarchy_Standard>
	\begin{itemize}
		\item /bin --- binaries essential for the system (used by everyone)
		\item /sbin --- binaries for managing the system (usually for sysadmin)
		\item /lib --- libraries used for /bin, /sbin
		\pause \item /usr/bin, /usr/sbin, /usr/lib --- same as
			above, but not system critical
		\pause \item /usr/include --- header files (.h)
		\pause \item /tmp --- temporary storage (cleaned up at boot)
		\item /dev --- device files (tty0, lp0, hda)
		\item /mnt --- other filesystems (another partition, hard drive, CD-rom)
		\pause \item /home --- home directories of users \\
			\pause \quad /home/user1 \\
			       \quad /home/user2 \\
			       \quad \ldots \\
	\end{itemize}
\end{frame}

\begin{frame}
	\frametitle{. and ..}
	\begin{itemize}
		\item each directory has two filenames: \texttt{.}
			and \texttt{..}
		\pause \item \texttt{.} is the current directory
		\pause \item \texttt{..} is the directory above the current directory
	\end{itemize}

	\pause e.g. \texttt{/usr/lib/.} is another name for \texttt{/usr/lib}

	\pause and \texttt{/usr/lib/..} is another name for \texttt{/usr}
\end{frame}

\begin{frame}
	\frametitle{The Prompt}
	\begin{itemize}
		\item command line interface
		\item scriptable --- automation
		\item short commands for fast entry
		\item managed by a \emph{shell} \\
			\quad (on Linux, we'll be using a shell called \texttt{bash})
		\pause\item bash has tab-complete \pause --- please use it :-)
	\end{itemize}
\end{frame}
%}}}
\section{Filesystem commands}\frame{\insertsection}
%% echo {{{
\begin{frame}[fragile]
	\frametitle{echo}
	%\begin{lstlisting}
	%$ echo "Hello, world!"
	%Hello, world!
	%$
	%\end{lstlisting}
	
	echoes arguments to \texttt{stdout}

	this is built-in to bash, but \texttt{/bin/echo} also exists
\end{frame}
%}}}
% ls {{{
\begin{frame}[fragile]
	\frametitle{ls}
	\begin{lstlisting}[escapeinside=||]
	$ ls         # what's in the current directory?
	test.c      test.h
	note        contrib/
	|\pause|$ ls contrib # what's in contrib/ ?
	data.txt
	$
	\end{lstlisting}
	\pause
	list files and directories

	options:
	\begin{itemize}
		\item -a : all files (including hidden files)
		\item -l : long format (shows extra information)
	\end{itemize}
\end{frame}
%}}}
% hidden files {{{
\begin{frame}
	\frametitle{hidden files}
	In Unix, hidden files are any files that begin with a dot.

	e.g.
	\begin{itemize}
		\item .
		\item ..
		\item .bashrc
		\item .vimrc
		\item .git/
	\end{itemize}

	usually used for configuration
\end{frame}
%}}}
% cd, pwd {{{
\begin{frame}[fragile]
	\frametitle{cd and pwd}
	\begin{lstlisting}
	$ cd contrib  # change directory
	$ ls
	data.txt
	$ pwd         # path to working directory
	/home/user1/project/contrib
	$
	\end{lstlisting}
\end{frame}
%}}}
% mkdir {{{
\begin{frame}[fragile]
	\frametitle{mkdir}
	\begin{lstlisting}[escapeinside=||]
	$ cd ..  # go back up to /home/user1/project
	|\pause|$ mkdir data
	$ cd data
	|\pause|$ ls
	$ # it's empty
	\end{lstlisting}

	make a directory

	\pause
	options:
	\begin{itemize}
		\item -p : make any parents as well\\
			\qquad\texttt{mkdir -p a/b/c}
	\end{itemize}
\end{frame}
%}}}
% touch, redirection, cat {{{
\begin{frame}[fragile]
	\frametitle{touch, redirection, cat}
	\begin{lstlisting}[escapeinside=||]
	$ touch more_data.txt
	|\pause|$ ls
	more_data.txt
	|\pause|$ cat more_data.txt # new file is empty
	|\pause|$ echo "Some data" > more_data.txt
	|\pause|$ cat more_data.txt
	Some data
	|\pause|$ echo "Extra data" > extra_data.txt
	|\pause|$ ls
	more_data.txt      extra_data.txt
	\end{lstlisting}

	\pause \texttt{touch} creates a file if it doesn't exist or updates timestamp of file

	\pause\texttt{cat} outputs contents of files \pause (Be careful with
	binary files. They can mess up your terminal. Use
	\texttt{reset} to fix that.)
\end{frame}
%}}}
% cp, mv {{{
\begin{frame}[fragile]
	\frametitle{cp, mv}
	\begin{lstlisting}[escapeinside=||]
	$ cp more_data.txt more_data.txt.old
	# copy file
	|\pause|$ ls
	more_data.txt   extra_data.txt    more_data.txt.old
	|\pause|$ mv extra_data.txt more_data.txt.old
	# move/rename file
	|\pause|$ ls
	more_data.txt   more_data.txt.old
	\end{lstlisting}

	\pause
	options for \texttt{cp}:
	\begin{itemize}
		\item -p : preserve timestamp
		\item -r, -R : recursively copy (needed for directories)
	\end{itemize}

	My own habit for copying directories is to use \texttt{cp~-puvR}.
\end{frame}
%}}}
% rm {{{
\begin{frame}[fragile]
	\frametitle{rm}
	\begin{lstlisting}[escapeinside=||]
	$ cd ..
	|\pause|$ ls
	test.c      test.h
	note        contrib/
	data/
	|\pause|$ rm note
	|\pause|$ ls
	test.c      test.h
	contrib/    data/
	\end{lstlisting}
\end{frame}
\begin{frame}[fragile]
	\frametitle{rm}
	\begin{lstlisting}[escapeinside=||]
	|\pause|$ rm data/*
	# use wildcard to remove all files inside
	$ ls data
	$ rmdir data # remove empty directory
	$ ls
	test.c      test.h     contrib/
	|\pause|$ rm -R contrib
	|\pause|$ ls
	test.c      test.h
	\end{lstlisting}

	\pause
	options for \texttt{rm}:
	\begin{itemize}
		\item -r, -R : recursively delete (needed for directories)
	\end{itemize}

	I use \texttt{-R} instead of \texttt{-r} because it stands
	out more. There is no built-in trash can in Unix.
\end{frame}
%}}}
% du, df {{{
\begin{frame}[fragile]
	\frametitle{du, df}
	\begin{lstlisting}[escapeinside=||]
	$ du
	[|\ldots| outputs file sizes
	 for the current directory |\ldots| ]
	$ df
	[|\ldots| outputs disk info |\ldots| ]
	\end{lstlisting}
\end{frame}
%}}}

\section{Documentation}\frame{\insertsection}
% man pages {{{
\begin{frame}[fragile]
	\frametitle{man}
	\begin{lstlisting}[escapeinside=||]
	$ man ls
	[ documentation for ls ]
	$ man printf
	[ documentation for the printf command ]
	$ man 3 printf
	[ documentation for printf() in stdio.h ]
	$ man man
	\end{lstlisting}
	\pause
	man pages have a standard layout to make navigating them
	easier

	\pause Search man pages uses \texttt{apropos}.

	\pause Some commands are shell built-ins. Use \texttt{help} to
	see the documentation for these
\end{frame}
%}}}
% conventions {{{
\begin{frame}[fragile]
	\frametitle{Conventions}
	\begin{lstlisting}[escapeinside=||]
	$ ls -a    # short option
	|\pause|$ ls -la   # short options can be combined
	|\pause|$ ls -l -a # or separate
	|\pause|$ ls --all # long options have two hyphens
	|\pause|$ ls --all --full-time # must be separate
	\end{lstlisting}

	Most commands display their usage when you use a help
	option: \texttt{--help}, \texttt{-h}, or \texttt{-?}
\end{frame}
%}}}

\section{Access control}\frame{\insertsection}%{{{
\begin{frame}
	\frametitle{Users}
	\begin{itemize}
		\item root user (admin)
		\pause\item su, sudo (change user)
		\pause\item whoami (who is the current user?)
		\pause\item w  (who is logged in?)
	\end{itemize}
\end{frame}
\begin{frame}
	\frametitle{Groups}
	\begin{itemize}
		\item \texttt{groups} (get the groups of a user)
	\end{itemize}
\end{frame}
\begin{frame}[fragile]
	\frametitle{Permissions}
	\begin{itemize}
		\item Each file is owned by a single user and
			a single group
		\pause\item shown in \texttt{ls~-l} listing
		\pause\item we can set the group that a file belongs to
		\begin{lstlisting}[escapeinside=||]
		$ chgrp mygroup project.txt
		\end{lstlisting}
	\end{itemize}
\end{frame}
\begin{frame}
	\frametitle{Permissions}
	\begin{itemize}
		\item Three kinds of permissions
			\begin{itemize}
				\item r --- read
				\item w --- write
				\item x --- execute
			\end{itemize}
		\pause\item we can set these settings for the
			owner of the file, the group the file is
			in, and anyone else
	\end{itemize}
\end{frame}
\begin{frame}[fragile]
	\frametitle{Permissions}
	\begin{lstlisting}[escapeinside=||]
	$ ls -l this_needs_to_execute
	-r-------- 1 user1 group1  this_needs_to_execute
	|\pause|$ chmod u+x this_needs_to_execute  # user
	|\pause|$ chmod g+rx this_needs_to_execute # group
	|\pause|$ chmod o+rx this_needs_to_execute # other
	|\pause|$ ls -l this_needs_to_execute
	-r-xr-xr-x 1 user1 group1  this_needs_to_execute
	\end{lstlisting}
\end{frame}
%}}}


\begin{frame}
	\frametitle{Next time}
	\begin{itemize}
		\item processes
		\item more on I/O redirection
		\item screen (terminal multiplexer)
		\item advanced scripting
		\item network
	\end{itemize}
\end{frame}


\end{document}
